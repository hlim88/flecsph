\documentclass[notes.tex]{subfiles}
 
\begin{document}


\section{Applications}

\subsection{Sod Shock Tube}
The Sod shock tube is the test consists of a one-dimensional Riemann problem with the following initial parameters
\begin{equation}
(\rho, v, p)_{t=0} = \begin{cases}
(1.0,0.0,1.0) & \text{if} \indent 0 < x \leq 0.5 \\
(0.125,0.0,0.1) & \text{if} \indent 0.5 < x < 1.0
\end{cases}
\end{equation}
This link shows some references and values for Sod shock tube problem that also includes boundary and jump conditions(\url{http://www.phys.lsu.edu/~tohline/PHYS7412/sod.html}).

Also, we would like to re-generate the shock test result from Rosswog's paper. In that paper, he shows the result of a 2D relativistic shock tube test where the left state is given by $[P, v_x, v_y, N]_L = [40/3,0,0,10]$ and the right state by $[P, v_x, v_y, N]_R = [10^{-6},0,0,1]$ with $\Gamma = 5/3$

In our code, we use below parameters to get results

\subsection{Sedov Blast Wave}
A blast wave is the pressure and flow resulting from the deposition of a large amount of energy in a small very localized volume. This is another great test problem for computational fluid dynamics field.

There are different version of blast wave test but we consider the analytic solution for a point explosion is given by Sedov, making the assumption that the atmospheric pressure relative to the pressure insider the explosion negligible. The position of the shock as a function of time $t$, relative to the initiation of the explosion, is given by
\begin{equation}
R(t) = \left( \frac{e t^2}{\rho_0} \right)^{\frac{1}{\delta+2}}
\end{equation}
with $\delta = 2$ and $\delta = 3$ for cylindrical and spherical geometry respectively. The initial density $\rho_0$ whereas $e$ is a dimensionless energy. Right behind the shock we ahve the following properties
\begin{align}
\rho_2 = \frac{\Gamma +1}{\Gamma-1} \rho_0
P_2 = \frac{2}{\Gamma+1} \rho_0 w^2
v_2 = \frac{2}{\Gamma+1} w
\end{align}
where the shock velocity is
\begin{equation}
w(t) = \frac{d R}{dt} = \frac{2}{\delta+2} \frac{R(t)}{t}
\end{equation}

In numerical simulations, energy deposition in a single point is difficult to achieve. A solution to the problem is to make use of the bursting balloon analogue. Rather than depositing the total energy in a single point, the energy is released into a balloon of finite volume $V$
\begin{equation}
e = \frac{(P-P_0)V}{\Gamma -1}
\end{equation}
The energy release in a balloon of radius $r_0$ raises the pressure to the value
\begin{equation}
P = \frac{3(\Gamma-1)e}{(\delta+1) \pi r_0^{\delta}}
\end{equation}
Here, we test 2D blast wave test. In this simulation, we use ideal gas EOS with $\Gamma = 5/3$ and we are assuming that the undistributed area is at rest with a pressure $P_0 = 1.0 \time 10^{-5}$. The density is constant $\rho_0$, also in the pressurized region.


\subsection{Equations of State}
\label{sec:eos}
To understand the inner property of stars, one needs to find the equation which describes the relation between the pressure of matter and its density, temperature and other compositions such that
\begin{equation}
P = P(\rho, T, Y_e, ...)
\end{equation}
First, we consider analytic equations of state that are relevant for binary neutron stars
\subsubsection{Ideal Gas}
Ideal gas equation of state is
\begin{equation}
P(\rho,u) = (\Gamma - 1) \rho u
\end{equation}
where $\Gamma$ is the adiabatic index of the gas. For a monatomic gas, we set $\Gamma = 5/3$. For another test such as sod tube, people use $\Gamma = 1.4$
\subsubsection{Piecewise Polytrope}
For more relevant simulation, we choose piecewise polytropic EOS.(Ideal gas EOS is still good for many simple test cases like Sod tube) In our case, we assume constant entropy so that many thermodynamic situations can be approximated as polytropes or piecewise functions made up of polytropes.

For neutron star case, we assume degenerated Fermi gas of neutrons then polytropic constant for a non-relativistic degenerated neutron gas is
\begin{equation}
K_0 = \frac{(3 \pi^2)^{2/3} \hbar^2}{5 m_n^{8/3}}
\end{equation}
where $m_n$ is the mass of a proton and $\hbar$ is a Planck constant. For polytropic index, we set
\begin{equation}
\gamma_0 = \frac{5}{3}
\end{equation}
In the relativistic case,
\begin{equation}
\gamma_1 = \frac{5}{2}
\end{equation}
Then, piecewise polytrope EOS is
\begin{equation}
P(\rho) = \begin{cases}
K_0 \rho^{\gamma_0} & \text{if} \indent \rho \leq \rho_0 \\
\frac{K_0 \rho_0^{\gamma_0}}{\rho_0^{\gamma_1}} \rho^{\gamma_1} & \text{if} \indent \rho > \rho_0
\end{cases}
\end{equation}
where $\rho_0 = 5 \times 10^{14} g/cm^3$. We can combine the piecewise polytropic EOS with ideal gas to attain an EOS valid at both low and high densities. For more realistic studies, we need to consider different types of analytic EOSs such as Maxwell-Boltzmann and Helmholtz EOSs. Also, we will put the functionality that can control tabulated EOS.

\subsubsection{Zero Temperature Equations of State}
Another interesting problem using SPH is the double white dwarf (DWD) simulations for studying possible progenitors to type $\Romannum{1}$a supernovae. Here, we use zero temperature equations of state (ZTWD) as a variation of the self consistent field technique. In ZTWD, the electron degeneracy pressure $P$ varies with the mass density $\rho$ according to the relation
\begin{equation}
P = A \left[ x(2x^2-3)(x^2+1)^{1/2} + 3 \sinh^{-1} x \right]
\end{equation}

where the dimensionless parameter
\begin{equation}
x \equiv \left( \frac{\rho}{B} \right)^{1/3}
\end{equation}
and the constant A and B are
\begin{align}
A \equiv \frac{\pi m_e^4 c^5}{3h^3} = 6.00288 \times 10^{22} \, \text{dynes} \, \text{cm}^{-2} \\
\frac{B}{\mu_e} \equiv \frac{8 \pi m_p}{3} \left(\frac{m_e c}{h} \right)^3 = 9.81011 \times 10^5 \, \text{g} \, \text{cm}^{-3}
\end{align}


\end{document}