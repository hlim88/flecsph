\documentclass[notes.tex]{subfiles}
 
\begin{document}

\section{Kernels}
\label{sec:kernels}
There are many choices for selecting a smoothing kernel. In the code, the
kernel choice is specified by a {\tt sph\_kernel} parameter, and the smoothing
length is defined to be equal to the kernel support radius. Note that some
works define smoothing length differently, such that the kernel support radius
becomes a multiple of smoothing length. For example, for a cubic spline kernel
it is twice the $h$. We do not make such distinction; all kernels below must
satisfy the following normalization condition:
\begin{align}
  \iiint_{S_D(h)} W(\vec{r},h) d^D\vec{r} = 1,
\label{eq:kernel_normalization_condition}
\end{align}
where the integration is performed over the $D$-dimensional volume of a sphere
$S_D(h)$ of radius $h$.

All the currently implemented kernels possess spherical symmetry, which makes
it easy to impose exact conservation of linear momentum. For such kernels, the
gradients can be computed as follows:
\begin{align}
  \nabla_a W_{ab} &\equiv \nabla_a W(|\vec{r}_a - \vec{r}_b|, h_a) \\
                  &= \frac{dW}{dr} \vec{\varepsilon}_{ab},
\end{align}
where $\vec{\varepsilon}_{ab}\equiv\vec{r}_{ab}/|\vec{r}_{ab}|$ is a unit
vector in the direction from particle $b$ to particle $a$.
It is therefore sufficient to implement $dW/dr$ for each kernel.

In the formulae below, we define $q\equiv|\vec{r}|/h$.

% Cubic Spline 
\subsection{{\tt sph\_kernel = "cubic spline"}}
The simplest (but \emph{not} the best one) is a Monaghan's cubic spline kernel:
\begin{equation}
W(\vec{r},h) = \frac{\sigma_D}{h^D} 
  \begin{cases}
    1 - 6 q^2 + 6 q^3 & \text{if} \; 0 \leq q \leq 1/2, \\
    2(1 - q)^3        & \text{if} \; 1/2 \leq q \leq 1, \\
    0                 & \text{otherwise},
\end{cases}
\end{equation}
where $q = r/h$, $D$ is the number of dimensions and 
$\sigma_D$ is a normalization constant:
\begin{equation}
\sigma_D =  \left\{ \frac43, \frac{40}{7\pi},\frac8{\pi}\right\}
         \; \text{in 1D, 2D and 3D resp.}
\end{equation}

Radial derivative of the cubic spline kernel is:
\begin{align}
 \frac{dW}{dr} = \frac{\sigma_D}{h^{D+1}} 
  \begin{cases}
   -6q(2 - 3q)  & \text{if} \; 0 \leq q \leq 1/2, \\
   -6(1 - q)^2  & \text{if} \; 1/2 \leq q \leq 1, \\
    0           & \text{otherwise}.
\end{cases}
\end{align}

%Gaussian
\subsection{{\tt sph\_kernel = "gaussian"}}
The Gaussian kernel is:
\begin{equation}
W(\vec{r},h) = \frac{\sigma_D}{h^D} 
  \begin{cases}
    e^{-q^2} & \text{if} \; 0 \leq q \leq 3, \\
    0        & \text{if} \;  q >3 
\end{cases}
\end{equation}
where $\sigma_D$ is :
\begin{equation}
\sigma_D =  \left\{ \frac{1}{\pi^{1/2}}, \frac{1}{\pi},\frac{1}{\pi^{3/2}}\right\}
         \; \text{in 1D, 2D and 3D resp.}
\end{equation}

Radial derivative of the cubic spline kernel is:
\begin{align}
 \frac{dW}{dr} = \frac{\sigma_D}{h^{D+1}} 
   \begin{cases}
    -2 q e^{-q^2} & \text{if} \; 0 \leq q \leq 3, \\
    0        & \text{if} \;  q >3 
\end{cases}
\end{align}

%Quintic
\subsection{{\tt sph\_kernel = "quintic spline"}}
The Quintic spline kernel is:
\begin{equation}
W(\vec{r},h) = \frac{\sigma_D}{h^D} 
  \begin{cases}
    [(3-q)^5-6(2-q)^5+15(1-q)^5] & \text{if} \; 0 \leq q \leq 1, \\
    [(3-q)^5 - 6(2-q)^5] & \text{if} \; 1 \leq q \leq 2, \\
    (3-q)^5 & \text{if} \; 2 \leq q \leq 3, \\
    0        & \text{if} \;  q >3 
\end{cases}
\end{equation}
where $\sigma_D$ is:
\begin{equation}
\sigma_D =  \left\{ \frac{1}{120}, \frac{7}{478\pi}, \frac{3}{359\pi}\right\}
         \; \text{in 1D, 2D and 3D resp.}
\end{equation}

Radial derivative of the cubic spline kernel is:
\begin{align}
 \frac{dW}{dr} = \frac{\sigma_D}{h^{D+1}} 
  \begin{cases}
    [-5(3-q)^4+30(2-q)^4-75(1-q)^4] & \text{if} \; 0 \leq q \leq 1, \\
    [-5(3-q)^4 + 30(2-q)^4] & \text{if} \; 1 \leq q \leq 2, \\
   -5(3-q)^4 & \text{if} \; 2 \leq q \leq 3, \\
    0        & \text{if} \;  q >3 
\end{cases}
\end{align}


%Wendland
\subsection{{\tt sph\_kernel = "Wendland C2"}}
The Wendland C2-continuous kernel (C2) for 2D and 3D is
\begin{equation}
W(\vec{r},h) = \frac{\sigma_D}{h^D} 
  \begin{cases}
    \left(1-\frac{q}{2} \right)^4 (2q+1) & \text{if} \; 0 \leq q \leq 2, \\
    0        & \text{if} \;  q > 2 
\end{cases}
\end{equation}
where $\sigma_D$ is:
\begin{equation}
\sigma_D =  \left\{ \frac{7}{4 \pi}, \frac{21}{16\pi}\right\}
         \; \text{in 2D and 3D resp.}
\end{equation}
For 1D:
\begin{equation}
W(\vec{r},h) = \frac{\sigma_D}{h} 
  \begin{cases}
    \left(1-\frac{q}{2} \right)^3 (1.5q+1) & \text{if} \; 0 \leq q \leq 2, \\
    0        & \text{if} \;  q > 2 
\end{cases}
\end{equation}
where $\sigma_D$ is:
\begin{equation}
\sigma_D =  \frac{5}{8}
         \; \text{in 1D}
\end{equation}


% Kernels that will be added into FleCSPH
There are more kernels that we are adding to the code. Below is several examples
\subsection{Super Gaussian Kernel}
\begin{equation}
W(\vec{r},h) = \frac{1}{\pi^{D/2} h^D} 
  \begin{cases}
    e^{-q^2} \left(\frac{D}{2} + 1 - q^2 \right) & \text{if} \; 0 \leq q \leq 3, \\
    0        & \text{if} \;  q >3 
    \end{cases}
\end{equation}

\subsection{Wendland C4 Kernel}

\subsection{Wendland C6 Kernel}

\subsection{Sinc Kernel}

We also refer \href{https://pdfs.semanticscholar.org/6ae2/960b7cbeab3e1969033b343dbe3594c99cb3.pdf}{Liu 2010} for more discussions.

%%% Using this, we can calculate kernel gradient. Below procedure shows the steps of kernel gradient in 3D
%%% $$W = \frac{1}{\pi h^3} \times \begin{cases} 1 - \frac{3}{2} (\frac{r}{h})^2 + \frac{3}{4} (\frac{r}{h})^3, & \mbox{si } 0 \leq \frac{r}{h} < 1 \\ \frac{1}{4} [2-\frac{r}{h}]^3, & \mbox{si } 1 \leq \frac{r}{h} < 2\\ 0, & \mbox{si } \frac{r}{h} \geq 2 \end{cases}$$
%%% 
%%% And $r=\sqrt{(x_i-x_j)^2 + (y_j-y_j)^2 + (z_i-z_j)^2}$
%%% with $r = \sqrt{u^2+v^2+w^2}$ and
%%% $ \vec{r_{ij}} = \begin{cases} u = x_i - x_j \\ v = y_i-y_j \\ w = z_i-z_j \end{cases} $
%%% 
%%% $$ \vec{\nabla} . W = \frac{\partial W}{\partial u} \vec{x} + \frac{\partial W}{\partial v} \vec{y} + \frac{\partial W}{\partial w} \vec{z}  =  \frac{\partial W}{\partial r} \frac{\partial r}{\partial u} \vec{x} + \frac{\partial W}{\partial r}  \frac{\partial r}{\partial v} \vec{y} + \frac{\partial W}{\partial r} \frac{\partial r}{\partial w} \vec{z} $$
%%% $$ = \frac{\partial W}{\partial r} \Big[ \frac{\partial r}{\partial u}\vec{x} + \frac{\partial r}{\partial v} \vec{y} + \frac{\partial r}{\partial w} \vec{z}\Big] = \frac{\partial W}{\partial r} \frac{\vec{r_{ij}}}{r} $$
%%% 
%%% For $0 \leq \frac{r}{h} < 1$ :
%%% $$ \frac{\partial W}{\partial r} = - \frac{3}{h^2}r + \frac{9}{4h^3}r^2$$
%%% 
%%% $$ \vec{\nabla}_iW(\vec{r_{ij}},h) =\Big( -\frac{3}{h^2}r + \frac{9}{4h^3}r^2\Big) \frac{\vec{r_{ij}}}{r} = \Big( -\frac{3}{h^2} + \frac{9}{4h^3}r\Big) \vec{r_{ij}} $$
%%% 
%%% For $1 \leq \frac{r}{h} < 2$ :
%%% $$ \frac{\partial W}{\partial r} = \frac{-3}{4h} \Big(2-\frac{r}{h}\Big)^2 = \frac{-3}{4h} \Big( 4 - \frac{4r}{h} + \frac{r^2}{h^2} \Big) = \frac{-3}{h} + \frac{3r}{h^2} + \frac{-3r^2}{4h^3}$$
%%% 
%%% $$ \vec{\nabla}_iW(\vec{r_{ij}},h) =\Big(\frac{-3}{h} + \frac{3r}{h^2} + \frac{-3r^2}{4h^3} \Big) \frac{\vec{r_{ij}}}{r} = \Big(\frac{-3}{hr} + \frac{3}{h^2} + \frac{-3r}{4h^3} \Big) \vec{r_{ij}}  $$
%%% 
%%% So:
%%% 
%%% $$
%%%  \vec{\nabla}_iW(\vec{r_{ij}},h) = \frac{1}{\pi h^4} \times \begin{cases} ( -\frac{3}{h} + \frac{9}{4h^2}r) \vec{r_{ij}}, & \mbox{si } 0 \leq \frac{r}{h} < 1 \\ (\frac{-3}{r} + \frac{3}{h} + \frac{-3r}{4h^2} ) \vec{r_{ij}}, & \mbox{si } 1 \leq \frac{r}{h} < 2\\ 0, & \mbox{si } \frac{r}{h} \geq 2 \end{cases}$$




\end{document}
